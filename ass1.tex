\documentclass{unswmaths}
\usepackage{unswshortcuts}
\begin{document}


\subject{Complex Analysis}
\author{Edward McDonald}
\title{Assignment 1}
\studentno{3375335}
\setlength\parindent{0pt}
\unswtitle{}


\newcommand{\Real}{\operatorname{Re}}

\section*{Question 1}
For this question, define the function
\begin{equation*}
    \zeta(z) = \sum_{n=1}^\infty n^{-z}.
\end{equation*}
\begin{lemma}
    $\zeta$ is defined and complex differentiable on the domain $\{z \in \Cplx\;:\;\Real(z) > 1\}$
\end{lemma}

Now define the function $\zeta_1(z) = $...
\begin{lemma}
    $\zeta_1(z) = \zeta(z)$ for all $\Real(z) > 1$.
\end{lemma}
\begin{theorem}
    $\zeta_1$ is defined for $z \in \{z\;:\;\Real(z) > 0, z\neq 1\}$.
\end{theorem}

\section*{Question 2}
For this question, $f_1,f_2:\Cplx\rightarrow\Cplx$ are complex differentiable
with $f_2 \neq 0$.
\begin{theorem}
    The function $f = f_1/f_2$ cannot have infinitely many poles within the unit ball.
\end{theorem}
\begin{proof}
    If $f$ is undefined precisely when $f_2 = 0$, so the poles
    of $f$ within the unit ball must be zeros of $f_2$ within the unit ball.
    Assume that $f$ has infinitely many poles in the unit ball.
    
    However, since $f_2 \neq 0$, the zeros of $f_2$ are a discrete set.
    
    By the Bolzano-Weierstrass theorem, any infinite set in the open unit ball
    has a limit point in the closed unit ball. Hence if $f_2$
    has infinitely many zeros inside the unit ball, the zeros of $f_2$
    are not discrete and this is a contradiction.
\end{proof}

\section*{Question 3}
\begin{theorem}
    The function
    \begin{equation*}
        f(z) = \sqrt{a^2-1}\sin\left(\frac{\pi}{2K(k)}\int_0^{\frac{z}{\sqrt{k}}\right)\frac{1}{\sqrt{(1-t^2)(1-k^2t^2)}}dt\right)
    \end{equation*}
    maps the the unit ball conformally to the interior of an ellipse
\end{theorem}

\section*{Question 4}



\end{document}
